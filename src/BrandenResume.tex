\documentclass{ExpressiveResume}

% ============= PACKAGES ==============

\usepackage{hyperref}

% ============= PACKAGES ==============

\begin{document}

% You will write your resume here.

\resumeheader[
    firstname=Branden,             % Your first name
    lastname=Kim,              % Your last name
    email=brandenkiminq@gmail.com,                 % Your email
    phone=703-223-1009,                 % Your phone number, formatted as XXX-XXX-XXXX
    linkedin=branden-kim-17704513b,              % Your LinkedIn handle (without the @)
    github=psiionik,                % Your GitHub handle (without the @)
    city=Fairfax,                  % Your city of residence (ignored if no `state` is given)
    state=,                 % Your state of residence
    qrcode=./images/qr.png,                % the path to a qr code to show in the top right corner
]

\objective{
    Senior Software Engineer with 5 years of experience, pursuing a
    masters degree in ML part-time seeking opportunities to develop
    data pipelines for ML infrastructure.
}

% \summary{
%     % Write your summary statement here.
% }

% ----- Work Experience -----
\section{Work Experience}

\experience{Sonos}{
    \role{Senior Software Engineer}{Oct 2023  - Feb 2025}{
        \achievement{
            Led development of a global build and release pipeline delivering firmware to 50M+ devices.
            \begin{itemize}
                \item Reduced release time by 80\% by optimizing pipeline stages and storage.
            \end{itemize}
        }
        \achievement{
            Architected a distributed Update Server that aggregates
            update information from products and delivers firmware
            updates to 50 million products worldwide, reducing costs from
            \$700,000 to \$10,000.
            \begin{itemize}
                \item Built Redis caching and CDN edge-layer to handle 88\% of traffic, improving latency by 300\%.

                \item Deployed via Kubernetes with high availability and blue-green CI/CD using GitHub Actions + Jenkins.

                \item Built data streaming pipeline with
                      S3, Kinesis, and RDS to store 3M users and 50M
                      product data.
            \end{itemize}
        }
        \achievement{
            Created a functional, strongly-typed, end-to-end data pipeline
            enabling declarative virtual transaction chains and centralizing
            all developer and release firmware build metadata.
            \begin{itemize}
                \item Reduced firmware build-to-release cycle from 6h to 2h.

                \item Built type-safe framework for declarative SQL
                      composition with atomic transactions.
            \end{itemize}
        }
        \achievement{
            Developed a Scala-based CI/CD pipeline framework
            automating firmware promotion (Alpha → Prod).
            \begin{itemize}
                \item Virtualized Jenkins jobs, cutting plugin-related bugs by 72\%.
            \end{itemize}
        }
    }
    \role{Software Engineer}{Aug 2020 - Oct 2023}{
        \achievement{
            Built a fullstack microservices app with AWS Lambda, Step Functions,
            S3, and Serverless
            for internal CRUD operations on users, products, and updates.
            \begin{itemize}
                \item Built using custom MapReduce with Step Functions +
                      S3 and enabled batch user updates.
                      % \item Implemented CI/CD with Jenkins, GitHub Actions, and
                      %       Gitflow for efficient parallel development and feature rollout.
            \end{itemize}
        }

        \achievement{
            Built internal observability stack (Prometheus + Loki)
            aggregating Jenkins metrics,
            improving test reliability by 12\%.
        }

        \achievement{
            Automated firmware delivery workflows with Python scripts
            and Jenkins job optimizations.
            % \begin{itemize}
            %     \item Cut average runtime from 12 mins → 1.5 mins using
            %           Python multiprocessing.
            % \end{itemize}
        }
    }
}

% ----- Projects -----
\section{Technical Projects}

\begin{itemize}
    \item \tech{scwab} → Built a compiler for a C-style language
          (wabbit) using Scala and functional programming design.
    \item \tech{Transformer-Based Electronic Sub-Genre Classifier} →
          Built custom Transformer model for electronic subgenre
          classification using BERT + Fourier transforms.
    \item \tech{Java Raytracer} → Developed 3D raytracer using Java +
          Processing: ray generation, BVH, Phong shading, geometry collisions.
\end{itemize}

% ----- Education -----
\section{Education}

\degree{M.S Computer Science ML Specialization}{Georgia Institute
    of Technology}{2021 - Present}{
}

\degree{B.S Computer Science \honors{Summa Cum Laude}}{University of
    Virginia}{2016 - 2020}{
    % Extra information, e.g. \achievement's
}

\degree{Bradfield School of Computer Science}{Certifcation of Completion}{2022 - 2023}{
}

% ----- Tools / Skills -----
\section{Skills}

\tech{Languages:} Python, TypeScript, JavaScript, Java, Scala, Go, C

\tech{Infra/DevOps:} Docker, Kubernetes, Terraform, Jenkins, GitHub
Actions

\tech{Cloud:} AWS (Lambda, Step Functions, S3, Kinesis, RDS), Serverless

\tech{Databases:} Postgres, Redis

\tech{ML Tools:} PyTorch, NumPy, Pandas

\end{document}